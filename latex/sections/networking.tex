\section{Networking}

\begin{frame}
  \frametitle{It's almost as simple}
  \begin{itemize}
    \item{OOTB Support for ICMP, TCP, UDP, IPv4/6, Multicast}\pause{}
    \item{Includes support for resolving addresses}\pause{}
    \item{Also support TLS through OpenSSL}\pause{}
    \item{``Byte-based'' and ``Condition-based''}
  \end{itemize}
\end{frame}

\begin{frame}
  \frametitle{An echo server with:}
  \begin{itemize}
    \item{N connections}\pause{}
    \item{Output queueing}\pause{}
    \item{Automatic connection lifetime}
  \end{itemize}
\end{frame}

\begin{frame}[fragile]
  \frametitle{The connection class (head)}
  \begin{minted}[fontsize=\footnotesize{}]{cpp}
  struct connection : std::enable_shared_from_this<connection> {
    // ... members
  };
  \end{minted}
\end{frame}

\begin{frame}[fragile]
  \frametitle{The connection class (data members)}
  \begin{minted}[fontsize=\footnotesize{}]{cpp}
  struct connection : ... {

  private:
    asio::ip::tcp::socket m_sock;
    asio::strand m_strand;
    std::array<char, 1024> m_buff{};
    std::deque<std::string> m_out{};
  };
  \end{minted}
\end{frame}

\begin{frame}[fragile]
  \frametitle{The connection class (ctor)}
  \begin{minted}[fontsize=\footnotesize{}]{cpp}
  struct connection : ... {

    connection(asio::ip::tcp::socket sock)
      : m_sock{std::move(sock)}
      , m_strand{m_sock.get_io_service()} {

    }

  };
  \end{minted}
\end{frame}

\begin{frame}[fragile]
  \frametitle{The connection class (starting communication)}
  \begin{minted}[fontsize=\footnotesize{}]{cpp}
  struct connection : ... {

    void start() {
      do_read();
    }

  };
  \end{minted}
\end{frame}

\begin{frame}[fragile]
  \frametitle{The connection class (reading data)}
  \begin{minted}[fontsize=\footnotesize{}]{cpp}
  struct connection : ... {

  private:
    void do_read() {
      m_sock.async_read_some(asio::buffer(m_buff),
        [&, self = shared_from_this()](auto error, auto read) {
        if(!error) {
          auto begin = m_buff.data();
          auto end = begin + read;
          m_strand.post([&, data=std::string{begin, end}]{
            write(data);
          });
        do_read();
        }
      });
    }
  };
  \end{minted}
\end{frame}

\begin{frame}[fragile]
  \frametitle{The connection class (writing data part 1)}
  \begin{minted}[fontsize=\footnotesize{}]{cpp}
  struct connection : ... {

  private:
    void write(std::string data) {
      m_out.push_back(data);

      if(m_out.size() < 2) {
        do_write();
      }
    }
  };
  \end{minted}
\end{frame}

\begin{frame}[fragile]
  \frametitle{The connection class (writing data part 2)}
  \begin{minted}[fontsize=\footnotesize{}]{cpp}
  struct connection : ... {

  private:
    void do_write() {
      asio::async_write(m_sock, asio::buffer(m_out.front()),
        m_strand.wrap([&, self = shared_from_this()](auto error,
                                                    auto written) {
        m_out.pop_front();

        if(!error) {
          if(!m_out.empty()) {
            do_write();
          }
        }
      }));
    }
  };
  \end{minted}
\end{frame}

\begin{frame}
  \frametitle{Review of the connection class}
  Connections:\pause{}
  \begin{itemize}
    \item{keep themselves alive}\pause{}
    \item{have a single input buffer}\pause{}
    \item{have an output queue}
  \end{itemize}
\end{frame}

\begin{frame}[fragile]
  \frametitle{The server class (data members)}
  \begin{minted}[fontsize=\footnotesize{}]{cpp}
  struct server {

  private:
     asio::ip::tcp::acceptor m_acceptor;
     asio::ip::tcp::socket m_sock;
  };
  \end{minted}
\end{frame}

\begin{frame}[fragile]
  \frametitle{The server class (creation)}
  \begin{minted}[fontsize=\footnotesize{}]{cpp}
  struct server {

  server(asio::io_service & service)
    : m_acceptor{service,
                 asio::ip::tcp::endpoint{
                   asio::ip::tcp::v4(),
                   4321}}
    , m_sock{service} {}

  };
  \end{minted}
\end{frame}

\begin{frame}[fragile]
  \frametitle{The server class (starting)}
  \begin{minted}[fontsize=\footnotesize{}]{cpp}
  struct server {

  void start() {
    do_accept();
  }

  };
  \end{minted}
\end{frame}

\begin{frame}[fragile]
  \frametitle{The server class (accepting connections)}
  \begin{minted}[fontsize=\footnotesize{}]{cpp}
  struct server {

  private:
    void do_accept() {
      m_acceptor.async_accept(m_sock, [&](auto error){
        if(!error) {
          auto conn = std::make_shared<connection>(
            std::move(m_sock)
          );
          conn->start();
          do_accept();
        }
      });
    }
  };
  \end{minted}
\end{frame}

\begin{frame}
  \frametitle{Review of the server class}
  Our server:\pause{}
  \begin{itemize}
    \item{accepts new connections}\pause{}
    \item{creates \mintinline{cpp}{connection} objects}\pause{}
    \item{does not care about the connections}
  \end{itemize}
\end{frame}

\begin{frame}[fragile]
  \frametitle{Glueing it all together}
  \begin{minted}[fontsize=\footnotesize{}]{cpp}
    auto && service = asio::io_service{};
    auto && server = ::server{service};

    server.start();
  \end{minted}
\end{frame}

\begin{frame}[fragile]
  \frametitle{Glueing it all together (contd.)}
  \begin{minted}[fontsize=\footnotesize{}]{cpp}
    service.run();
  \end{minted}
\end{frame}

\begin{frame}[fragile]
  \frametitle{Glueing it all together (contd.)}
  \begin{minted}[fontsize=\footnotesize{}]{cpp}
    auto const cpus = std::thread::hardware_concurrency();
    auto pool = std::vector<std::future<void>>{cpus};

    for(auto & future : pool) {
      future = std::async(std::launch::async, [&]{
        service.run();
      });
    }
  \end{minted}
\end{frame}

\begin{frame}
  \begin{center}
    \Huge{Demo Time!}
  \end{center}
\end{frame}

\begin{frame}
  \frametitle{There is more}
  \begin{itemize}
    \item{stream buffers}\pause{}
    \item{conditions}\pause{}
    \item{TLS}\pause{}
    \item{Name resolution}\pause{}
    \item{Synchronous I/O}
  \end{itemize}
\end{frame}
