\section{ASIO 101}

\begin{frame}
  \frametitle{Overview}
  \begin{itemize}
    \item{Created by Christopher Kohlhoff in 2003}\pause{}
    \item{Part of the Boost libraries since 2008}\pause{}
    \item{Available on Linux, BSD, macOS, Windows, \ldots{}}\pause{}
    \item{Based on the Proactor pattern}\pause{}
    \item{NOT only for asynchronous I/O}
  \end{itemize}
\end{frame}

\begin{frame}[fragile]
  \frametitle{ASIO Hello World}
  \begin{minted}[fontsize=\footnotesize{},autogobble]{cpp}
  #include <asio.hpp>
  #include <chrono>
  #include <iostream>

  int main() {
    auto && service = asio::io_service{};
    auto && timer = asio::steady_timer{service};

    timer.expires_from_now(std::chrono::seconds{1});
    timer.async_wait([](auto error){
      std::cout << "timer fired!\n";
    });

  service.run();
  }
  \end{minted}
\end{frame}

\begin{frame}[fragile]
  \frametitle{Disecting the Example}
  \begin{itemize}
    \item{Our Proactor}
    \item{\mintinline{cpp}{service = asio::io_service{};}}\pause{}
    \item{The I/O object}
    \item{\mintinline{cpp}{timer = asio::steady_timer{service};}}\pause{}
    \item{Our asynchronous operation}
    \item{\mintinline{cpp}{timer.async_wait(...)}}\pause{}
    \item{Our completion handler}
    \item{\mintinline{cpp}{[](auto error){ ... }}}
  \end{itemize}
\end{frame}

\begin{frame}[fragile]
  \begin{centering}
    \Large{This is the basic anatomy of an ASIO program}
  \end{centering}
\end{frame}
